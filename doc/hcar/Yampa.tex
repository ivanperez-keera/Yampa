% Yampa.tex
\begin{hcarentry}[section,updated]{Yampa}
\report{Ivan Perez}%11/15
\label{yampa}
\makeheader

% \documentclass{article}
% \usepackage{hyperref}
% \usepackage{graphicx}
% 
% \begin{document}

% \begin{center}
% \includegraphics[width=.4\linewidth]{kslogo512}
% \end{center}

Yampa (Github: \href{http://git.io/vTvxQ}{http://git.io/vTvxQ}, Hackage:
\href{http://goo.gl/JGwycF}{http://goo.gl/JGwycF}), is a Functional Reactive
Programming implementation in the form of a EDSL to define \emph{Signal
Functions}, that is, transformations of input signals into output signals (aka.
\emph{behaviours} in other FRP dialects). 

Yampa systems are defined as combinations of Signal Functions. The core of
Yampa includes combinators to create constant signals, apply  pointwise (or
time-wise) functions to signals, access the running time of a signal function,
introduce delays and create loopbacks (carrying present output as future
input). These systems can also be dynamic: their structure can change by using
\emph{switching} combinators, which enable the application of a different
signal function at some point in the execution. Combined with combinators to
deal with signal function collections, this enables a form of dynamic FRP in
which new signals can be introduced, frozen, unfrozen, removed and altered at
will.

Yampa is designed to guarantee \emph{causality}: the value of an output signal
at a time $t$ can only depend on values of input signals at times $[0,t]$.
Yampa restricts access to other signals only to the immediate past, by letting
signals functions carry \emph{state} for the future.  FRP signal functions
implement the Arrow and ArrowLoop typeclasses, making it possible to
use both the arrow notation and arrow combinators. A suitable thinking model
for FRP in Yampa is that of signal processing, in which components (signal
functions) transform signals based on their present value and the component's
internal state. Components can be serialized, applied in parallel, etc.

Unlike other implementations of FRP, Yampa enforces a strict separation of
effects and pure transformations. All IO code must exist outside the Signal
Functions, making Yampa systems easier to reason about and debug.

Yampa has been used to create both free/open-source and commercial games.
Examples of the former include Frag (\href{http://goo.gl/8bfSmz}{http://goo.gl/8bfSmz}), a basic
reimplementation of the Quake III Arena engine in Haskell, and Haskanoid
(\href{http://git.io/v8eq3}{http://git.io/v8eq3}), an arkanoid game featuring
SDL graphics and sound with Wiimote \& Kinect support. Examples of the latter
include Keera Studios' Magic Cookies!
(\href{https://goo.gl/0A8z6i}{https://goo.gl/0A8z6i}), a board game for Android
written in Haskell and avaliable via Google Play for Android store.

\begin{figure}[h]
\begin{center}
\includegraphics[width=\linewidth]{androidbreakout}
\caption*{\textit{Haskanoid, a Yampa game with Kinect and Wiimote support,
running on an Android tablet.}}
\end{center}
\end{figure}

% Yampa can be used to express simple physics in a very concise manner.
% Examples can be seen at \href{http://git.io/v8esW}{http://git.io/v8esW}
% (Youtube: https://www.youtube.com/watch?v=GOBhhtxfhi0), and and 

Yampa is actively maitained. The last updates have focused on introducing
documentation, structuring the code to facilitate navigation, eliminating
legacy code superceeded by other Haskell libraries, and increasing code quality
in general. Over the years, performance in FRP has been an active topic of
discussion and Yampa has been optimised heavily (games like Haskanoid have been
clocked at over 700 frames per second on a standard PC). Also because Yampa is
\emph{pure}, the introduction of parallelism is straightforward. In future
versions, the benchmarking package \texttt{criterion} will be used to evaluate
and increase performance. We encourage all Haskellers to participate by opening
issues on our Github page (\href{http://git.io/vTvxQ}{http://git.io/vTvxQ}),
adding improvements, creating tutorials and examples, and using Yampa in their
next amazing Haskell games.

Extensions to Arrowized Functional Reactive Programming are an active research
topic. The Functional Programming Laboratory at the University of Nottingham is
working on several extensions to make Yampa more general and modular,
facilitate other uses cases, increase performance and work around existing
limitations. To collaborate with our research on FRP, please contact Ivan Perez
at \href{mailto:ixp@cs.nott.ac.uk}{ixp@cs.nott.ac.uk} and Henrik
Nilsson at \href{mailto:nhn@cs.nott.ac.uk}{nhn@cs.nott.ac.uk}.


\end{hcarentry}
